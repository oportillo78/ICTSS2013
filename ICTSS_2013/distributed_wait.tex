
%%%%%%%%%%%%%%%%%%%%%%%%%%%%%%%%%%%%%%%%%%%%%%%%%%%%%%%%%%%%%%%%%%%%%%%%%%%%%%%%%%%%%%%%%%%%%%%%%%%%%%%%%%%%
% Author: Omar Portillo
%%%%%%%%%%%%%%%%%%%%%%%%%%%%%%%%%%%%%%%%%%%%%%%%%%%%%%%%%%%%%%%%%%%%%%%%%%%%%%%%%%%%%%%%%%%%%%%%%%%%%%%%%%%%

%%%%%%%%%%%%%%%%%%%%%%%%%%%%%%%%%%%%%%%%%%%%%%%%%%%%%%%%%%%%%%%%%%%%%%%%%%%%%%%%%%%%%%%%%%%%%%%%%%%%%%%%%%%%
% Defining document class + packages to be used
%%%%%%%%%%%%%%%%%%%%%%%%%%%%%%%%%%%%%%%%%%%%%%%%%%%%%%%%%%%%%%%%%%%%%%%%%%%%%%%%%%%%%%%%%%%%%%%%%%%%%%%%%%%%

\documentclass[runningheads,a4paper]{llncs}

\usepackage{amssymb}
\setcounter{tocdepth}{3}
\usepackage{graphicx}
\usepackage{epstopdf}

\usepackage{url}
\newcommand{\keywords}[1]{\par\addvspace\baselineskip
\noindent\keywordname\enspace\ignorespaces#1}

\begin{document}

%%%%%%%%%%%%%%%%%%%%%%%%%%%%%%%%%%%%%%%%%%%%%%%%%%%%%%%%%%%%%%%%%%%%%%%%%%%%%%%%%%%%%%%%%%%%%%%%%%%%%%%%%%%%
% TITLE + AUTHORS' INFORMATION
%%%%%%%%%%%%%%%%%%%%%%%%%%%%%%%%%%%%%%%%%%%%%%%%%%%%%%%%%%%%%%%%%%%%%%%%%%%%%%%%%%%%%%%%%%%%%%%%%%%%%%%%%%%%

\mainmatter  % start of an individual contribution

% first the title is needed
\title{Pending to define title:\\.. Pending to define title}

% a short form should be given in case it is too long for the running head
\titlerunning{Pending to define title}

% the name(s) of the author(s) follow(s) next
%
% NB: Chinese authors should write their first names(s) in front of
% their surnames. This ensures that the names appear correctly in
% the running heads and the author index.
%
\author{Andres-Omar Portillo-Dominguez\inst{1}\and Miao Wang\inst{1}\and Philip
Perry\inst{1} \and\\
John Murphy\inst{1} \and Nick Mitchell\inst{2}\and Peter F. Sweeney\inst{2}}

\authorrunning{Pending to define title}
% (feature abused for this document to repeat the title also on left hand pages)

\urldef{\mailucdconnect}\path|andres.portillo-dominguez@ucdconnect.ie,|
\urldef{\mailucd}\path|{philip.perry,miao.wang,j.murphy}@ucd.ie,|
\urldef{\mailibm}\path|{nickm,pfs}@us.ibm.com|

% the affiliations are given next; don't give your e-mail address
% unless you accept that it will be published
\institute{Lero - The Irish Software Engineering Research Centre, Performance\\
Engineering Laboratory, UCD School of Computer Science and Informatics,
University College Dublin, Ireland\\
\mailucdconnect\\
\mailucd\\
%\url{http://www.ucd.ie/}
\and
IBM T.J. Watson Research Center,\\
Yorktown Heights, New York, USA\\
\mailibm}%\\
%\url{http://www.ibm.com/}


\toctitle{Lecture Notes in Computer Science}
\tocauthor{Authors' Instructions}
\maketitle

%%%%%%%%%%%%%%%%%%%%%%%%%%%%%%%%%%%%%%%%%%%%%%%%%%%%%%%%%%%%%%%%%%%%%%%%%%%%%%%%%%%%%%%%%%%%%%%%%%%%%%%%%%%%
% Abstract 
%%%%%%%%%%%%%%%%%%%%%%%%%%%%%%%%%%%%%%%%%%%%%%%%%%%%%%%%%%%%%%%%%%%%%%%%%%%%%%%%%%%%%%%%%%%%%%%%%%%%%%%%%%%%

\begin{abstract}
[[PENDING: The abstract should summarize the contents of the paper and
should contain at least 70 and at most 150 words. It should be written using the
\emph{abstract} environment]].
\keywords{[[PENDING: performance testing, automation, performance analysis,
system monitoring]]}
\end{abstract}

examples: Creating fine tuned and stable systems is very
important and requires use of a list of testing tools that analyze various resources (like GC logs, heapdumps, native memory, etc). Due to the nature of those tools, this kind of analysis can only be performed by a small group of expert users that have high technical skills. In this paper we present an approach for expert tool development in the field of performance testing. The result of this approach is the creation of GcLite tool, an expert tool for analyzing garbage collection logs. A case study was carried out in a real industry environment showing the benefits to a number of testing teams. The benefit of the tool is that it allows a wider range of testers to carry out expert analysis.

<<
Our approach relies on a lightweight error detection mechanism based on the idea 
of replaying test executions against the model. We further show how the
error detection capabilities can be integrated into a convincing argument
for tool qualification, going through the necessary verification activities
step-by-step. We highlight the key steps for the RT-Tester Model-Based
Test Generator, which is used in test campaigns in the automotive, railway
and avionic domains. The approach avoids having to qualify several
complex components present in model-based testing tools, such as code
generators for test procedures and constraint solving algorithms for test
data elaboration.

Performance is a pervasive quality of software
systems; everything affects it, from the software itself
to all underlying layers, such as operating system,
middleware, hardware, communication networks, etc.
Software Performance Engineering encompasses
efforts to describe and improve performance, with two
distinct approaches: an early-cycle predictive modelbased
approach, and a late-cycle measurement-based
approach. Current progress and future trends within
these two approaches are described, with a tendency
(and a need) for them to converge, in order to cover
the entire development cycle.
>>

%%%%%%%%%%%%%%%%%%%%%%%%%%%%%%%%%%%%%%%%%%%%%%%%%%%%%%%%%%%%%%%%%%%%%%%%%%%%%%%%%%%%%%%%%%%%%%%%%%%%%%%%%%%%
% Introduction
%%%%%%%%%%%%%%%%%%%%%%%%%%%%%%%%%%%%%%%%%%%%%%%%%%%%%%%%%%%%%%%%%%%%%%%%%%%%%%%%%%%%%%%%%%%%%%%%%%%%%%%%%%%%


\section{Introduction}

When doing performance testing, the detection of performance related issues and the identification of their root causes are challenging and time consuming tasks, especially in highly distributed environments.

WAIT (idle-time analysis) has proven successful in simplifying the detection of performance related issues and their root causes in Java environments.

The main strengths of WAIT makes it an attractive candidate to the performance testing domain, as it would have minimal impact in the testing results:
- WAIT uses a very light-weight approach which does not require any instrumentation/changes to the monitored environment. 
- It also has very low overhead.

Despite its strengths, WAIT has certain scalability limitations that prevent its effective usage in the performance testing domain:
- Manual data collection and uploading per monitored process.
- Lack of synchronization with load testing tools. 
- Lack of capability to perform periodic data refresh during the test execution to have incremental results.
Even though the above limitations might be manageable in small testing environments or test runs, they prevent WAIT to be effectively use in bigger testing environments (precisely the scenario where its performance analysis capabilities would be more valuable).
The identified adoption barriers are the result of recurrent discussions with our industrial partner, which practically make WAIT non-scalable, in its current form, in highly distributed environments.

Other ideas:
Problem to solve: Hard to detect and identify the root cause of performance
related issues in a multi-node environment. Hard to use WAIT in these circumstances.

Benefits: Less Time (setup, monitor, analysis, go/no-go decision) -> Less
Learning curve / Less Error-Prone, Share knowledge / experience (remain), 
“Real-time”, Keep as light-weight as possible (minimum add-ons + low overhead)

Qualifiers: Testing Domain, Distributed env, Using Available info

---

•	In more detail/ additionally supported by the needs of our industrial partner:
o	Manual data collection.
o	Uploading too.
o	Benchmark / Overhead.
o	Aggregated results.
o	Real time / Incremental.
•	Important to rephrase it as estimated tester time benefit? (at least a high level estimate … do not forget the time dimension!)

[PENDING: To review/copy papers: RTCE, Memory, GC Lite, WAIT, Observations of
Java App]


Summary of potential phrases:
•	This paper proposes …
•	Highlights of methodology.
•	“The results of this work have provided evidence regarding the feasibility of the proposed approach” + summary of key results.
•	“The main contributions of this paper are:”
o	Approach + implementation to automate the data gathering and execution of WAIT in sync with a performance test loader. “Generic” enough so that it can be “easily” enough to adapt it to other performance analysis tools and/or performance test loaders.
o	Keeping the main strengths of WAIT: low overhead + avoiding intrusion (under the scope of the paper).
o	Any qualitative insights? (i.e. javacore cost of the class loader section)

Main contributions:

1. Approach/Architecture + implementation to automate the data gathering and execution of WAIT in sync with a performance test loader (or make WAIT scalable / usable in a performance testing domain).
2. “Generic” enough solution so that it can be adaptable to other performance analysis tools and/or performance test loaders.
3. Keeping the main strengths of WAIT: low overhead + avoiding intrusion (under the scope of the paper).
4. Any qualitative insights? (i.e. javacore cost of the class loader section)
<<
– We introduce a lightweight framework for replaying concrete test executions
with the aim of identifying erroneous test case executions.

The contribution of this paper is an
approach to make the principle of parameterized unit testing available
to black-box GUI testing. The approach is based on the new notion of
parameterized GUI tests. We have implemented the approach in a new
tool. In order to evaluate whether parameterized GUI tests have the potential
to achieve high code coverage, we apply the tool to four open
source GUI applications. The results are encouraging.
>>

examples:

<<
Currently, there is a real need for enterprise applications to meet their 
performance requirements such that client requests are satisfied in a timely
manner. Delays in response time can lead to lost revenue for example in
e-commerce applications. Thus, being able to create and fine tune a stable
system is very important. For enterprise Java applications garbage collection logs, 
thread dumps, thread usage statistics, heap dumps, CPU utilization, JVM memory usage, 
JDBC pool status and server response time are only some of the resources a tester can use 
in order to understand how the application is performing. Unfortunately, often a deep 
technical system understanding (of the intervals of JVM, the application server, the 
J2EE technology, and in some cases the application source code) is required in order to 
fully understand the information contained in the aforementioned logs. System and 
performance testers may not necessarily have this technical background and identifying 
problems or fine tuning systems can be difficult in these circumstances.

A common issue with current performance tools is that the
produced output requires a user with expert knowledge in order to understand and analyze 
it (e.g. heap dump analysis where a heap dump may contain information on millions of 
instances of objects). In addition, these deep technical skills required to use such 
tools are usually held by a small number of testers within a given organisation. This can 
lead to issues whereby particular analysis activities can only be carried out by a small 
number of experts which can reduce the overall productivity of large testing teams where 
expertise is not evenly distributed. Moreover, current tools are sometimes difficult to 
understand because of the (i) large volume of the data presented, (ii) the bad layout of 
the data which often fails to highlight what data is important and (iii) the lack of 
additional context data which can be used to understand the data in the context of a 
particular situation (e.g. context information might indicate that an application is 
under heavy load). From working with a large number of industry test teams we have found 
that there is a real need to develop tools that can produce information which is 
consumable by both domain and non-domain experts, allowing for a wider number of testers 
to carry out complex system analysis.

Our proposed solution is to develop an expert tool that can

be used by a wider pool of testers where traditionally this work was done only
by a small number of experts. Our approach was to develop a tool that can produce information consumable by both domain experts and non experts. This approach provides a high level view for non experts allowing them to identify issues and a more detailed fine grained view for experts allowing for detailed root cause analysis of the issues. We followed a two step approach for developing suc

We have validated the tool through a case study that was carried out across 6
different test teams in IBM, showing how such a tool allowed for a larger number of testers to carry out garbage collection analysis and ultimately identify and raise a large number of bugs as part of the garbage collection analysis.

The paper is structured as follows. In the next section we explain the approach
for creating an expert tool. Section 3 details the motivation for creating a garbage 
collection analysis tool and how the approach was realised through the implementation of 
GcLite. In this section we also give design and implementation details. In section 4 we 
discuss how we have validated our approach through a case study carried out within IBM 
across 6 different test teams and in section 5 we give our conclusion.
>>

<<Miao:

To ensure the research work is helpful for solving real-life problems for software
industries, we have carried out regular frequent meetings with IBM System Verification 
Test (SVT) team managers and discussed how our current research work can improve their testing experiences by avoiding many manual tasks. We assessed the quality of our solution based on their feedback in relation to high system throughput, memory efficiency, and accuracy. Furthermore, we are also occasionally involved in IBM cross team collaborations to share our knowledge and discuss how our research algorithms can be used to solve different problems faced by other teams. In order to improve industrial testing experiences, we have made the decision to design and implement a lightweight solution with specific rules for functional and system testing teams to ease their troubleshooting tasks.

>>

<<
Limitation in testing process / Reduce Costs/Time
System Complexity

- According to Myers [1], software testing is a process, or a series of processes,
designed to make sure that computer code does what it was designed to do and that it
does not do anything unintended. Software testing has become essential to the
companies to ensure the product quality regardless of whether the development
methodology. Software testing automation is one of the main approaches that
have been applied to decrease testing costs and time while test automation
requires automated test execution and results verification [2].
[[
1. Myers, G.J.: The Art of Software Testing. Ed. John Wiley & Sons, Inc., Hoboken, New
Jersey. (2004)
2. Shahamiri, S.R.: Kadir, W.M.N.W.; Mohd-Hashim, S.Z.: A Comparative Study on
Automated Software Test Oracle Methods. In: ICSEA '09. Fourth International Conference,
pp. 140 – 145, (2009).
]]

- (automation) Although the approach is usable by the company’s current employees, it was
considered complex to apply and error prone. Doing each step by hand requires
too much attention and if any single mistake is committed the whole process is
compromised. Thus, the automation of the approach was considered key to its
adoption.

(Intro) Related Techniques: In the software development process, performance testing is 
commonly conducted to determine an application's performance as experienced by the user.
In this context, the focus is generally not on a single application session but
on the application in its entirety, i.e., one does not test a single, isolated re-
quest of a single user but rather the performance when many users interact with
the application simultaneously. Software testers commonly use techniques such
as load testing, stress testing, etc., to perform these kinds of tests [3]. There
is a variety of commercial products and free open-source tools available for the
task. For example, IBM's Rational Performance Tester [4] and HPs LoadRunner
[5] both do scalability testing by generating a real work load on the application.
Open-source tools such as Apache's JMeter [6] and Grinder [7] provide a similar
functionality.
Motivated by the large cost for commercial performance testing tools, Chen
et al. created Yet Another Performance Testing Framework [8]. It enables users
to create custom test programs which dene the business operations to be per-
formed during the test. Chen's framework then executes these tasks concurrently.
In [9], Zhang et al. present a cloud-based approach to performance testing of web
services. Their system provides a frontend in which users can specify test cases
which are then dispatched to Amazon EC2 [10] cloud instances for execution.
Similar to all the previous tools, their system is testing the performance under
concurrent user access to the system.
[[
3. Molyneaux, I.: The Art of Application Performance Testing. Volume 1. O'Reilly
Media (2009)
4. IBM: Rational Performance Tester. http://www-01.ibm.com/software/
awdtools/tester/performance/
5. Hewlett Packard: HP LoadRunner. http://www8.hp.com/us/en/software/
software-product.html?compURI=tcm:245-935779
6. Apache Software Foundation: Apache JMeter. http://jmeter.apache.org/
7. Aston, P.: The Grinder, a Java Load Testing Framework. http://grinder.
sourceforge.net/
8. Chen, S., Moreland, D., Nepal, S., Zic, J.: Yet Another Performance Testing
Framework. In: Australian Conference on Software Engineering (ASWEC). (2008)
170 \{179
9. Zhang, L., Chen, Y., Tang, F., Ao, X.: Design and Implementation of Cloud-based
Performance Testing System for Web Services. In: Conference on Communications
and Networking in China (CHINACOM). (2011) 875 \{880
10. Amazon Web Services LLC: Amazon Elastic Compute Cloud (Amazon EC2).
http://aws.amazon.com/ec2/ (2012)
]]

Software performance (considered here as
concerned with capacity and timeliness) is a pervasive
quality difficult to understand, because it is affected by
every aspect of the design, code, and execution
environment. By conventional wisdom performance is
a serious problem in a significant fraction of projects.
It causes delays, cost overruns, failures on deployment,
and even abandonment of projects, but such failures
are seldom documented. A recent survey of
information technology executives [15] found that half
of them had encountered performance problems with at
least 20% of the applications they deployed.

[15] Compuware, Applied Performance Management Survey, Oct. 2006.

Like other software engineering activities, SPE is
constrained by tight project schedules, poorly defined
requirements, and over-optimism about meeting them.
Nonetheless adequate performance is essential for
product success, making SPE a foundation discipline in
software practice.

A determining factor for performance is that
resources have a limited capacity, so they can
potentially halt/delay the execution of competing users
by denying permission to proceed. Quantifying such
effects is an important task of SPE.

Lessons from the current work. There are many
weaknesses in current performance processes. They
require heavy effort, which limits what can be
attempted. Measurements lack standards; those that
record the application and execution context (e.g. the
class of user) require source-code access and
instrumentation and interfere with system operation.
There is a semantic gap between performance concerns
and functional concerns, which prevents many
developers from addressing performance at all. For the
same reason many developers do not trust or
understand performance models, even if such models
are available. Performance modeling is effective but it
is often costly; models are approximate, they leave out
detail that may be important, and are difficult to
validate.

The survey of information technology executives
[15], which found that half of them had had
performance problems with at least 20% of the
applications they deployed, commented that many
problems seem to come from lack of coverage in
performance testing, and from depending on customers
to do performance testing in the field.

Significant process issues are unresolved. Detail in
measurement and modeling must be managed and
adapted. Excessive detail is expensive and creates
information overload, while insufficient detail may
miss the key factor in a problem. Information has to be
thrown away. Models and measurements are discarded
even though they possibly hold information of longterm
value.

Source: Future Performance Engineering

Today, the state of industrial
performance measurement and testing techniques is
captured in a series of articles by Scott Barber [8][9]
including the problems of planning, execution,
instrumentation and interpretation.

[8] S. Barber, “Creating Effective Load Models for
Performance Testing with Incomplete Empirical Data”, in
Proc. 6th IEEE Int. Workshop on Web Site Evolution, 2004,
pp. 51-59.
[9] S. Barber, “Beyond performance testing”, parts 1-14,
IBM DeveloperWorks, Rational Technical Library, 2004,
www-128.ibm.com/developerworks/rational/library/4169.html

The tools used by performance analysts range from
load generators, for supplying the workload to a
system under test, to monitors, for gathering data as the
system executes.

Performance engineering is gaining attention, as
companies discover to their detriment that the
performance of their applications is often below
expectations. In the past, these problems were not
found until very late in the development of a product as
performance validation, if any, was one of the last
activities done before releasing the software. With
agile processes, the problem is unchanged if not worse
[11]. Thus early warning of performance problems is
still the challenge for SPE.

[11] S. Barber, “Tester PI: Performance Investigator”,
Better Software, March 2006, pp 20 – 25.

Developers and testers use instrumentation tools to
help them find problems with systems. However, users
depend on experience to use the results, and this
experience needs to be codified and incorporated into
tools. Better methods and tools for interpreting the
results and diagnosing performance problems are a
future goal.

Technical developments -> Visualization and diagnostics

Understanding the source of performance limitations
is a search process, which depends on patterns and
relationships in performance data, often revealed by
visualizations. Promising areas for the future include
better visualizations, deep catalogues of performancerelated
patterns of behaviour and structure, and
algorithms for automated search and diagnosis.
Present visualization approaches use generic dataexploration
views such as Kiviat graphs (e.g. in
Paradyn [47]), traffic intensity patterns overlaid on
physical structure maps [47], CPU loading overlaid on
scenarios, and breakdowns of delay [66]. Innovative
views are possible. For example, in [60] all kinds of
resources (not just the CPU) are monitored, with tools
to group resources and focus the view. The challenge
for the future is to visualize the causal interaction of
behaviour and resources, rather than to focus on just
one or the other.

[[
[47] Merson, P. and Hissam, S. “Predictability by
construction” Posters of OOPSLA 2005, pp 134-135, San
Diego, ACM Press, Oct. 2005.
[60] J.A. Rolia, L. Cherkasova, R. Friedrich, “Performance
Engineering for EA Systems in Next Generation Data
Centres”, Proc. 6th Int. Workshop on Software and
Performance, Buenos Aires, Feb. 2007.
[66] C.U. Smith, L. G. Williams, Performance Solutions,
Addison-Wesley, 2002
]]


Technical developments -> Bottleneck identification

a search for a saturated
resource which limits the system, is a frequent
operation. In [21], Franks et al describe a search
strategy over a model, guided by its structure and
results, and detects under-provisioned resource pools
and over-long holding times. It combines properties of
resources and behaviour, for a “nested” use of
resources. It scales to high complexity, is capable of
partial automation, and could be adapted to interpret
measured data. A multistep performance enhancement
study using these principles is described in [83].
Another search strategy purely over data ([9], part 7)
focuses on reproducing and simplifying the conditions
in which the problem is observed. The actual diagnosis
of the cause (e.g. a memory leak) depends on designer
expertise.
A bottleneck search strategy combining the data and
the model could detect more kinds of problems (e.g.
both memory leaks and resource problems) and could
provide automated search assistance.
Patterns (or anti-patterns) related to bottlenecks have
been described by Smith and Williams [66] and others
(e.g., excessive dynamic allocation, “one-lane bridge”).
For the future, more patterns and more kinds of
patterns (on measurements, on scenarios or traces) will
be important. Patterns that combine design, model and
measurement will be more powerful than those based
on a single source.

[[
[9] S. Barber, “Beyond performance testing”, parts 1-14,
IBM DeveloperWorks, Rational Technical Library, 2004,
www-128.ibm.com/developerworks/rational/library/4169.html

[21] G. Franks, D.C. Petriu, M. Woodside, J. Xu, P.
Tregunno, “Layered Bottlenecks and Their Mitigation”, Proc
3rd Int. Conf. on Quantitative Evaluation of Systems,
Riverside, CA, Sept. 2006.

[83] J. Xu, M. Woodside, and D.C. Petriu, “Performance
Analysis of a Software Design using the UML Profile for
Schedulability, Performance and Time”, in Proc. 13th Int.
Conf. Modeling Techniques and Tools for Computer
Performance Evaluation, Urbana, USA, Sept. 2003
]]

Scalability analysis and improvement is largely a
matter of identifying bottlenecks that emerge as scale
is increased, and evolving the architecture. Future
scalability tools will employ advanced bottleneck
analysis but will depend more heavily on models, since
they deal with potential systems.

Technical developments -> More efficient testing

Efficient testing covers the operational profile and
the resources of interest with minimum effort needed
to give sufficient accuracy. Accuracy is an issue
because statistical results may require long runs, and it
can be affected by other factors in the test design such
as the load intensity and the patterns of request classes
used. For example, systems under heavy load show
high variance in their measurements, which contributes
to inaccurate statistical results. It may be more fruitful
to identify the heavily-used resources at moderate
loads and (for purposes apart from stress testing) use
the results to extrapolate to heavy loads using a
performance model.

More effort is also required in performance testing
tools. The lack of standards for tool interoperation
increases the effort to gather and interpret data, and
reduces data-availability for new platforms.
Lightweight and automated instrumentation are old
goals that will continue to demand attention. Load
drivers are at present well-developed in commercial
tools, but more open tool development could speed
progress.


...Further automation
of data collection and better methods for deducing
causality look promising. More powerful and general
approaches to problem diagnosis are necessary and
possible....
>>

%%%%%%%%%%%%%%%%%%%%%%%%%%%%%%%%%%%%%%%%%%%%%%%%%%%%%%%%%%%%%%%%%%%%%%%%%%%%%%%%%%%%%%%%%%%%%%%%%%%%%%%%%%%%
% Background
%%%%%%%%%%%%%%%%%%%%%%%%%%%%%%%%%%%%%%%%%%%%%%%%%%%%%%%%%%%%%%%%%%%%%%%%%%%%%%%%%%%%%%%%%%%%%%%%%%%%%%%%%%%%

\section{Background}

•	Performance Testing Process

He uses different tools:
- Heap dumps: WAIT?
- Thread dumps: WAIT (locks, contentions) … around every 5 - 10 mins / ISA (performance analysis kit based on Eclipse).
- Overall development expertise (i.e. architecture, layers, frameworks, tools).
- Code profiling: tprof to profile in real time (recording every 1-2 mins to know the CPU usage per transactions); jprofiling (single transaction costs, such as CPU usage).
- Database: Tunning of SQL queries (or expert systems that encapsulate the identification of the most common issues).
- GC: gclite (expert system) or jmap.

Key message/conclusion: The identification of the root causes of performance issues, relies very heavy on the expert knowledge of the user + requires a lot of different tools. There is no possible to indicate an average time per issue identification, but overall is very challenging/time-consuming.


•	WAIT and/or javacores? (check my bookmarks)

- Distributed Environments? - Evaluate Cloud Distributed environment as possible focus/background info?



%%%%%%%%%%%%%%%%%%%%%%%%%%%%%%%%%%%%%%%%%%%%%%%%%%%%%%%%%%%%%%%%%%%%%%%%%%%%%%%%%%%%%%%%%%%%%%%%%%%%%%%%%%%%
% Related Work
%%%%%%%%%%%%%%%%%%%%%%%%%%%%%%%%%%%%%%%%%%%%%%%%%%%%%%%%%%%%%%%%%%%%%%%%%%%%%%%%%%%%%%%%%%%%%%%%%%%%%%%%%%%%

\section{Related Work}

Automated Performance Testing (support tools).
•	Performance Analysis Tools/Techniques (from WAIT papers).
•	WAIT

<<
Related Work: The idea of replaying a test execution in a simulator is, of course, not new. The
overall approach is frequently referred to as the capture and replay paradigm,
and has long been studied in different contexts such as testing of concurrent
programs [2]. ... To our best knowledge, our approach is the first to combine replay with model-based
methods for error detection within a test-case generator. Our contribution is not
a theoretical one, but comes from an industrial perspective.

Related Work: Unlike these approaches, our work aims at proposing a generic and platform
independent test system based on the TTCN-3 standard to execute runtime
tests. The proposed test system supports different test isolation mechanisms in
order to support testing different kinds of components: test sensitive, test aware
or even non testable components. Such test system has an important impact on
reducing the risk of interference between test behaviors and business behaviors
as well as avoiding overheads and burdens.

SPE, The commonest approach is purely measurement-based; it
applies testing, diagnosis and tuning late in the development cycle, 
when the system under development can be run and measured (see, e.g.
[2][4][8][9]).

[2] M. Arlitt, D. Krishnamurthy, J. Rolia, “Characterizing
the Scalability of a Large Web-based Shopping System'',
ACM Trans. on Internet Technology, v 1, 2001, pp. 44-69.

[4] A. Avritzer, J. Kondek, D. Liu, and E. J. Weyuker,
"Software performance testing based on workload
characterization," in Proc. WOSP’2002, Rome, , pp. 17-24.

[8] S. Barber, “Creating Effective Load Models for
Performance Testing with Incomplete Empirical Data”, in
Proc. 6th IEEE Int. Workshop on Web Site Evolution, 2004,
pp. 51-59.
[9] S. Barber, “Beyond performance testing”, parts 1-14,
IBM DeveloperWorks, Rational Technical Library, 2004,
www-128.ibm.com/developerworks/rational/library/4169.html

Performance testing on part or all of system, under
normal loads and stress loads [8]. The use of test data
to solve problems is the subject of [9]. This activity is
discussed in Section 3 below.

[8] S. Barber, “Creating Effective Load Models for
Performance Testing with Incomplete Empirical Data”, in
Proc. 6th IEEE Int. Workshop on Web Site Evolution, 2004,
pp. 51-59.
[9] S. Barber, “Beyond performance testing”, parts 1-14,
IBM DeveloperWorks, Rational Technical Library, 2004,
www-128.ibm.com/developerworks/rational/library/4169.html


•	Performance Testing Tools? (HP Load Runner, Jmeter, RPT).

[Pending to see where better to fit these points:]
Benefit: Tools usability, productivity increase (less time identifying issues and their root cause), any benefit in the consolidation (other than reducing from N*M reports to 1), i.e. trends in time and/or nodes (probably good input from performance team):
* How does the process currently work? (steps and time) i.e. ~2 hours debugging, 2-4 RCA + fixing OO
* Issues in different nodes sound like environmental ones?
* Issues in different times (kind of trending) might involve the application?
They refer to “end to end” testing … the most common issue are deadlocks.

>>

%%%%%%%%%%%%%%%%%%%%%%%%%%%%%%%%%%%%%%%%%%%%%%%%%%%%%%%%%%%%%%%%%%%%%%%%%%%%%%%%%%%%%%%%%%%%%%%%%%%%%%%%%%%%
% Problem Definition
%%%%%%%%%%%%%%%%%%%%%%%%%%%%%%%%%%%%%%%%%%%%%%%%%%%%%%%%%%%%%%%%%%%%%%%%%%%%%%%%%%%%%%%%%%%%%%%%%%%%%%%%%%%%

\section{Problem Definition}

PROBLEM DEFINITION: WAIT Challenges in Performance Testing environment 
XXX (Rephrase Industrial partner case)


%%%%%%%%%%%%%%%%%%%%%%%%%%%%%%%%%%%%%%%%%%%%%%%%%%%%%%%%%%%%%%%%%%%%%%%%%%%%%%%%%%%%%%%%%%%%%%%%%%%%%%%%%%%%
% Proposed Approach
%%%%%%%%%%%%%%%%%%%%%%%%%%%%%%%%%%%%%%%%%%%%%%%%%%%%%%%%%%%%%%%%%%%%%%%%%%%%%%%%%%%%%%%%%%%%%%%%%%%%%%%%%%%%

\section{Proposed Approach and Implementation}

<<
Approach and Implementation: In this section we present details of our approach and its 
implementation. As outlined in Section 1, there exist a bunch of issues in order to make the 
approach of parameterized GUI tests applicable to real world applications. Our
approach depicted in Figure 7 consists of the following consecutive steps: (1)
Event Flow Construction; (2) Symbolic Widget Injection; (3) Symbolic Event
Injection; (4) Event Handler Elevation; (5) Generation of Parameterized GUI
Tests, (6) Symbolic Execution, and (7) Replayer. + [Figure]


In this section we present our approach for creating an expert tool in the
field of performance testing. From a high level view as part of the approach for 
creating such a tool we need to (1) document expert knowledge (usually in the form 
of rules describing known issues) and then (2) apply the expert knowledge to flush out 
or identify issues that exist in the real system. Identifying issues in a real system 
using documented expert knowledge follows a four step approach previously outlined in 
the literature [9][10]. Next we describe how we applied this approach in the context of 
garbage collection analysis. The four main steps are (a) to monitor the system, (b) 
analyze the data collected, (c) detect issues within the analyzed data and (d) present 
any issues or potential issues identified to the end user.

Our approach relies on a lightweight error detection mechanism based on the idea 
of replaying test executions against the model. We further show how the
error detection capabilities can be integrated into a convincing argument
for tool qualification, going through the necessary verification activities
step-by-step. We highlight the key steps for the RT-Tester Model-Based
Test Generator, which is used in test campaigns in the automotive, railway
and avionic domains. The approach avoids having to qualify several
complex components present in model-based testing tools, such as code
generators for test procedures and constraint solving algorithms for test
data elaboration.
>>

1.	SCOPE LIMITS
o	Performance Testing, 
o	Web Applications

4.	ARCHITECTURE OVERVIEW/DESIGN
XXXOverview (similar visualization that I have presented before – but more polished-).
5.	ARCHITECTURE DETAILS …
XXX Explaining how the opportunity areas are addressed.

The GcLite tool is an expert tool for analyzing garbage
collection behavior. It is a Java application with an extensible design. Figure 1 shows a high level view of GcLite’s design and how it
enhances the design of current GC tools
[12][13][14].

1.	PROTOTYPE (CURRENT) IMPLEMENTATION
•	Describe RPT – WAIT + WebAgent; explaining the reason behind each main technical decision.

<<
The Run-Time Correlation Engine [4] (RTCE) has been developed in conjunction
with our industrial partner IBM, and serves us as a platform base for the symptom 
matching implementation presented in this paper. RTCE has been deployed by several test 
teams across the world for monitoring enterprise applications. RTCE can signifi- cantly 
save administrators’ time spent on system analysis [20]. For example, one particular 
testing team reports one full log analysis in 1 hour which is estimated to take up to 23 
working hours without automated support of RTCE.
>>

ANY LIMITATIONS?
RPT, but architecture generic enough (i.e. Web Agent)

<<
- (automation) A prototype tool is designed and developed based on the concepts presented here.
ANTLR parser generator is used to create syntax grammar for complete CICS based
COBOL language specifications. Similarly other parsers are also developed using
Java. Using the prototype tool developed, case studies have been conducted on an
industrial application. The application is legacy library management system running
in one of our client‟s library. The number of test cases generated for each of the three
modules is listed in Fig 3.a. A sample test scenarios generated is shown in Fig 3.b.

- (automation) Figure 2 shows the architecture of the tool we have implemented as a plug-in for
Eclipse [8], on top of EMF [11] and the UML2 plug-in. The functionality of this tool
includes . . .

Prototype: To realize trace verification, we have developed a passive testing tool [2], which
aims to automate the process of trace verification. A description of this tool is
given in Fig.3.
>>

%%%%%%%%%%%%%%%%%%%%%%%%%%%%%%%%%%%%%%%%%%%%%%%%%%%%%%%%%%%%%%%%%%%%%%%%%%%%%%%%%%%%%%%%%%%%%%%%%%%%%%%%%%%%
% Experimental Setup
%%%%%%%%%%%%%%%%%%%%%%%%%%%%%%%%%%%%%%%%%%%%%%%%%%%%%%%%%%%%%%%%%%%%%%%%%%%%%%%%%%%%%%%%%%%%%%%%%%%%%%%%%%%%

\section{Validation}

<<
Experiments: In this section we evaluate our approach. We compare how our approach performs,
(a) when the computation of input data is replaced by the use of random
values, and (b) when the Event Flow Graph is not considered for event sequence
generation. We first present the setup of the experiments. Then we discuss the
results of the experiments.
>>

2.	ENVIRONMENT? (VS. MERGING IT TO METHODOLOGY)
Plan A:
- Ideally, IBM Lotus Connections environment (multiple nodes, multiple PIDs).
- Main issue: Pending to know when will be ready (guesstimate is end of month)
Plan B / Backup plan: 
- Apache Web Server (load balancing) + 4 Tomcat (2 per node), using JPetStore
web app.
- Main issue: Overhead of installing + especially creating the RPT scripts!
(learning curve) In both scenarios:
- There will be a WAIT server + RPT node.
- There is available a single-node IBM Portal environment.

[PENDING: To summarize environment]

<<
Experimental Setup: Application Set. During the experimentation, three real world applications
were used:
- A
- B
>>

4.	METHODOLOGY

<<
We define the following two research questions: Q1 ... Q2 ...
Results of the Experiments: Figure 12 shows the results of the experiments. We answer Q1 with Yes ...
>>

Phase 1:
o	Objective: Validate WAIT Overhead vs. WAIT-RPT one.
o	Combinations (3):
	WAIT Presence (3): 
	Test without WAIT, 
	Test with WAIT (manual test), 
	Test with WAIT-RPT
	Web-Applications (1): 
	Portal (single node)
	Test configurations*:
	Workloads (1): To be defined (It should be big!) - 2,000
	Duration: 1 hour
	WAIT-RPT configuration:
	HD Threshold: Disabled (so that only the time threshold controls the process).
	Time Threshold: 10 minutes (Constant, so that 6 intervals occur).
	Sampling frequency: 8 minutes.* ----> 30 secs?
* For each identified combination, there will be 3 runs (~9 hours of total execution). WebAppServer restarted before every run.
NOTE: These configurations will be based on expert judgment of the SVT team (reflect “common” practice in the industry?)
Phase 2: 
o	Objective: Validate Architecture (qualitative, quantitative)
o	Combinations:
	WAIT Presence (2):
	Test run without WAIT,
	Test run with WAIT-RPT Integration
	Web Applications (1):
	Either Lotus Connection or JPetStore.
	Test configurations*:
	Workloads (1): To be defined (Tentatively 1, which should be big!) ~2,000 per node (~4,000 assuming 2 nodes)
	Duration: 24 hour (SVT uses 1 hr, 24 hour, 5 days, discarding 1 hr due to relevance and 5 days to do time constraints).
	WAIT-RPT configuration*:
	HD Threshold: 100 MB (pending to confirm).
	Time Threshold: 10 minutes (pending to confirm).
	Sampling frequency: 8 minutes.

For each identified combination, there will be 3 runs (~6 days of total execution). WebAppServer restarted before every run.
NOTE: These configurations will be based on expert judgment of the SVT team (reflect “common” practice in the industry?)
It might also help to show that the reliability of the solution (I might take a look to other “fancy” NFR-related words)

Phase 3: 
Use case … detection benefits of 1 report vs. many! … For the sake of progress, probably the last piece of the experiment!
Injecting defects, showing how they are better/more easily identified. PetStore.
Two dimensions:
- Time: Early detection or trends (stress early detection).
o	Nodes: Environmental issues.

Benefits of WAIT:
- Injecting bugs in JPetStore (Ideas from WAIT + SVT team).
- Benefit of single WAIT report, possible two dimensions:
- Time: Early detection (avoid wasting resources if serious issues identified).
- Nodes: Environmental issues (very likely cause of having an issue in some nodes and other not).
- Meeting to know the fixing side of the performance testing/analysis to get better insights about which use case would be preferrable.
- TEST METRICS: TESTER PRODUCTIVITY? (single case, but give an idea of the benefit)
http://www.softwaretestinggenius.com/articalDetails?qry=967

<<
The teams used performance and load testing tools like
LoadRunner [16] and IBM Rational Performance Tester [15]. They executed reliability runs 
(the ability of a system or component to perform its required functions under stated 
conditions for a specified period of time [24], usually 5 or 7 days) and performance runs 
(short period runs in order to measure something specific, like the response rate [25]).
 As The main aim of both performance and reliability testing
was to ensure that the applications had good performance (transaction and page response rates) even under heavy load, made reasonable usage of memory and didn’t have any memory leaks [31]. The performance of a software system has been described as an indicator of how well the system meets its requirements for timeliness [32]. Smith and Williams [32] describe timeliness as being measured in either response time or throughput, where response time is defined as the time required to respond to a request and throughput is defined as the number of requests that can be processed in some specific time interval. For

15: IBM. Rational Performance Tester. http://www-01.ibm.com/software/awdtools/tester/performance/ 
24: Wikipedia. Reliability engineering.
%http://en.wikipedia.org/wiki/Reliability_%28engineering%29
25: B.Subraya. Integrated Approach to Web Performance Testing: A Practitioner's Guide. IRM Press 2006

By the above case study we tried to see if our approach in
the creation of an expert tool in the field of garbage collection was successful. The measures of success were the ability to
identify bugs and problems that affect the applications as well as additional
test coverage by identifying new types of issues (e.g. finalizers, system.gc, etc). Moreover, the tool should allow a wider range of testers to perform the expert analysis, offering at the same time significant time savings for the expert users. 

The results from using the tool were very encouraging as
many bugs were identified as well as the time saved by using GcLite in some cases was 
quite high. In the table 2, we can see a summary of our case study. We can notice that 
the time saved per application is around 1.6 hours. GcLite’s automated collection and 
analysis of the GC logs, the ease of use, the layout of the output and the use of 
recommendations are responsible for the time savings. We also notice that an average of 
6.5 defects was identified using the tool.

Another important advantage of using the GcLite tool was
the ability to open precise bugs. Testers used the contextual information from the tool 
in order to investigate deeper the issues and provide detailed description in the SPRs 
that they were opening. Furthermore, less experienced testers used this information in 
order to gain a deeper understanding of several re-occurring issues.

In conclusion, the test case showed that the use of an expert
tool that uses the 4 step approach can have several advantages. The time saving from 
using GcLite automated collection of GC logs and analysis instead of another tool in 
some cases was quite high. This was also enhanced by the tool’s ease of use and the 
output’s ease to navigate. Moreover, the amount and variety of bugs that were identified 
shows that GcLite can analyze and discover issues within a GC log by using the 
recommendationEngine. 
>>

3.	KEY METRICS
- Qualitative vs. Quantitative results!
- Qualitative: Possible verbiage about how the detected adoption barriers are
addressed.
- Previously you would have ended with:
- X reports, assuming you monitored X nodes and generated all the reports after
the test execution finished.
- X*Y reports, assuming you monitored X nodes and generated the reports every
1/Y time.
- Now, in both cases you would have ended with a single report.
- Sanity checks: Not lost zip files, hard disk threshold respected.
- Quantitative: Overhead costs: (possible min, max, avg)
- Performance Testing: Response Time, Throughput
- Node Resource Usage: RAM, CPU


%%%%%%%%%%%%%%%%%%%%%%%%%%%%%%%%%%%%%%%%%%%%%%%%%%%%%%%%%%%%%%%%%%%%%%%%%%%%%%%%%%%%%%%%%%%%%%%%%%%%%%%%%%%%
% Section 5: Experimental Results
%%%%%%%%%%%%%%%%%%%%%%%%%%%%%%%%%%%%%%%%%%%%%%%%%%%%%%%%%%%%%%%%%%%%%%%%%%%%%%%%%%%%%%%%%%%%%%%%%%%%%%%%%%%%

\section{Experimental Results}


%[[PENDING: To append results\! (draft_paper_NEW and excel file) should include
%some tables, maybe a graphic or some type]]

1.	[PHASE 1]
o	Compare response time \& throughput with and without WAIT/WAIT-RPT.
o	Compare resource consumption per node with and without WAIT/WAIT-RPT.
+ Testing/Process metrics.
2.	[PHASE 2]
o	Compare response time \& throughput with and without WAIT-RPT.
o	Compare resource consumption per node with and without WAIT-RPT.
o	Validate that there was not “information lost” between the collection and uploading.

3.[PHASE 3]
X,Y to an actual number (use case) … more qualitative, but showing how it is best 1 than many!
Injecting 4 bugs, then run it using the same config than Phase 2 (2 in both nodes, 2 only on a specific one). Then two iterations will be run: 1st to catch up as many issues as possible. The 2nd one to see results after fixing issues (possible an intermediate run might be needed, as some issues might mask other ones) After each run, the detected bugs will be “fixed”.

<<
- (automation) This section presents a case study adapted from [8] and initially discussed 
in [3]. The SUT is a burglar alarm system whose goal is to monitor sensors to detect
the presence of intruders in a building. Consider a test case, presented in [3],
that covers the occurrence of an interruption (transfer to the backup power)
followed by the detection of an intruder finishing with a call to the police. The
objective of the case study presented in this paper is to show how to use the
developed API in test case automation (from step 2 to step 4 in Fig. 1). For this,
a version of the burglar alarm system was developed to run on the FreeRTOS
environment based on industrial PC (x86) port.

- This section summarizes a simplistic case study whose purpose is to illustrate
through a practical example the application of the principles described in Section 3. It
is worthy mentioning that rates have been included as constants declared using
reference values which are not shown. The actual values for these constants should be
changed as more detailed reliability data is available and has no influence on the
scope of this paper.

Empirical Results: In this section the experimental setup is presented, along with the workflow of
the experiments themselves. After that the experimental results are discussed.

>>

Z. THREADS TO VALIDITY:

•	Big variance in 24-hour runs … it seems to different behaviours: Week and
Weekend! (more runs). Environmental issues

<<
- Of course, the results observed cannot be generalised to other applications
different from the two case studies considered here. The presented case studies
are just meant to illustrate the differences that may arise. Wider experiments
need to be executed to get more general conclusions.

Threats to Validity: The main threat to the validity of these results is the fact that only 
three test subjects were used during the experimentation. Despite these subjects being
real world applications being diverse in both the size of the application (lines of
code) and size of the test suite, the limited number of subjects implies that not
all types of system’s are tested. This means that a system with characteristics
that are completly different might present different results.
Another threat is that the number of injected bugs is not enough to lead to
accurate results, as these bugs might simply be “lucky bugs” that intercept a
collar variable.
Naturally, there are also threats that are based on the implementation of the
invariants, the instrumentation or the pattern detector algorithms themselves.
The reduce these threats, additional testing was made prior to the experimentation
to guarantee the quality of the experimental results in this regard.

Threats to Validity: Beyond the selection bias due to the limited availability of open source 
C# applications, we report one threat to external validity: We evaluated four C# open
source applications which incorporate the Windows Forms toolkit for building
the GUI. Alternative programming languages and GUI toolkits, e.g., Java Swing,
follow different paradigms of building graphical user interfaces. For example, it
might be not possible to obtain event handlers during the construction of the
EFG. Thus, the construction of the EFG, the generation of parameterized GUI
tests, and the symbolic execution must be adapted to the corresponding environment.
In principle, there is no reason to believe that our approach is not
applicable to other environments.

>>

Statistic references:
Enable Add-on in Excel: http://click4biology.info/c4b/1/2007.htm
%Example in excel, pair t-test:
% http://www.stattutorials.com/EXCEL/EXCEL_TTEST2.html T-Distribution calculations in excel: https://courses.washington.edu/dphs568/course/excel-t.htm
Finding t critical value in excel: http://math.stackexchange.com/questions/10992/finding-critical-value-using-t-distribution-in-excel
how to use t-test in excel (example 2): http://blog.excelmasterseries.com/2010/08/how-to-use-t-test-in-excel-to-find-out.html
Other T-test reference: http://www.ruf.rice.edu/~bioslabs/tools/stats/ttest.html
Excel t-test function: http://www.excelfunctions.net/Excel-Ttest-Function.html
T-test Excel Help: http://office.microsoft.com/en-001/excel-help/ttest-HP005209325.aspx
Good example to learn T-test: http://www.aspfree.com/c/a/braindump/comparing-data-sets-using-statistical-analysis-in-excel/
---
%T-Test description: http://en.wikipedia.org/wiki/Student%27s_t-test
%Null Hypothesis description: http://en.wikipedia.org/wiki/Null_hypothesis
%Student T-Distribution description:
% http://en.wikipedia.org/wiki/Student%27s_t-distribution


%%%%%%%%%%%%%%%%%%%%%%%%%%%%%%%%%%%%%%%%%%%%%%%%%%%%%%%%%%%%%%%%%%%%%%%%%%%%%%%%%%%%%%%%%%%%%%%%%%%%%%%%%%%%
% Section 7: Conclusions
%%%%%%%%%%%%%%%%%%%%%%%%%%%%%%%%%%%%%%%%%%%%%%%%%%%%%%%%%%%%%%%%%%%%%%%%%%%%%%%%%%%%%%%%%%%%%%%%%%%%%%%%%%%%

\section{Conclusions and Future Work}

The objective of this work was to XXXX. To achieve this goal the 
paper presented a novel XXXX approach and its validation, which was composed of
a prototype and a study case. The results have proved that the 
proposed approach addresses the adoption barriers for WAIT in a
distributed performance testing environment from a qualitative perspective 
(desired behavior + minimum impact in the environment) and a quantitative 
(solution remains light weight and minimum overhead).

Furthermore, the approach can be easily extended to support other XXX tools?.

Future work will focus on three main venues:

• Assess how better to address the identified major overhead costs of using WAIT
in a distributed performance testing environment (i.e. more than 50\% of the
javacore information is not useful but still propagated from nodes to WAIT server. 
If removed it, the network overhead would go down by that percentage).

• Similarly, evaluate how best to exploit the information that can now be
obtained from a testing environment (i.e. test workload, response time, throughput and 
transactions) to improve the diagnosis quality and quantity of problems that WAIT can 
identify.

• More immediate is to evaluate the benefits of the proposed
architecture/approach in a bigger scenario, most likely through a study case of its 
adoption in our industrial partner SVT/IBM. 
/ 
Besides, we will concentrate on
further evidence of the practicability of our approach by applying it to additional case studies and industrial systems.

<<
Testing and tuning enterprise applications can be
challenging due to their complexity. It requires use of a list of testing tools from expert users that have high technical skills. 
In this paper we presented an approach for expert tool
development in the field of performance testing. 

The approach has been tested in a real industry environment and the results are encouraging for the tool and for the approach that we followed. 
The test case showed that the use of an expert tool that uses the 4 step approach can have several benefits to the user like time savings, opening 
precise bugs and supporting them with contextual information and educating less
experienced testers.

- (automation) This paper presents an ongoing work that addresses a solution to automation of
test case execution for RTES at the software and system integration level. The solution
provides an API to define PCOs such as the observation of values returned by functions, 
received messages, and timing associated with system responses.
A case study is also presented to show the applicability of the work. Though it
still needs to be extensively validated, the solution addresses the issues on test
execution raised in the introduction. It can deal with different development and
execution platforms once the SUT is instrumented to run on its target platform
and logs are generated to be evaluated at any environment. Moreover, the log
generation mechanism allows testing applications with hardware limitations.

Conclusion and Future Work: In this paper we have proposed a novel approach to the generation of 
GUI tests, implemented in a new tool called Gazoo. Gazoo selects event sequences
from the EFG of an application and generates a set of Parameterized GUI tests.
Then, Gazoo applies Pex in order to instantiate parameterized GUI tests. Finally,
Gazoo replays instantiated GUI tests on the application. In the terminology of
the black-box/white-box dichotomy, Gazoo starts with a black-box approach
(using the EFG in order to select executable test sequences), then moves on to a
white-box approach (in order to generate parameterized GUI tests and instantiate
them using Pex), and finally goes back to the black-box approach (using
a replayer in order to execute the (instantiated) GUI tests on the application).
As shown in the paper, we needed to overcome a number of non-trivial technical
hurdles in order to establish the appropriate interface between the black-box
approach and the white-box approach.
The scope of this paper was to show that our approach can achieve high
code coverage. Usually one expects that high code coverage translates to high
bug detection rate. For future work, we need to evaluate that this holds true
in our setting. This evaluation requires its own series of experiments where one
applies statistical methods to fault-seeded versions of AUTs, following, e.g., [11,
18].
Our work opens an interesting perspective for future research because the
general scheme behind our approach goes well beyond a specific tool, here Gazoo.
We need to explore different alternatives such as, e.g., [2, 16] and, e.g., [12, 19]
for going back and forth between the black-box approach and the white-box
approach in the sense described above.
>>

Example cite \cite{AbdelkaderLahmadi2005} blabla bla.


%%%%%%%%%%%%%%%%%%%%%%%%%%%%%%%%%%%%%%%%%%%%%%%%%%%%%%%%%%%%%%%%%%%%%%%%%%%%%%%%%%%%%%%%%%%%%%%%%%%%%%%%%%%%
% Section 8: Acknowledgements
%%%%%%%%%%%%%%%%%%%%%%%%%%%%%%%%%%%%%%%%%%%%%%%%%%%%%%%%%%%%%%%%%%%%%%%%%%%%%%%%%%%%%%%%%%%%%%%%%%%%%%%%%%%%

\subsubsection*{Acknowledgments}
This work was supported, in part, by Science Foundation Ireland grant 10/CE/I1855 to Lero - the Irish Software Engineering Research Centre (www.lero.ie).

\bibliographystyle{splncs}
\bibliography{dwait}

%\section*{Appendix: Springer-Author Discount}

\end{document}

Possible footnotes:
•	Windows Performance Metrics: http://technet.microsoft.com/en-us/library/cc768048.aspx
•	Javacore: http://publib.boulder.ibm.com/infocenter/javasdk/v1r4m2/index.jsp?topic=%2Fcom.ibm.java.doc.diagnostics.142%2Fhtml%2Fjavadumpsum.html
•	RPT: http://publib.boulder.ibm.com/infocenter/rpthelp/v8r2m1/index.jsp
•	WAIT: http://wait.ibm.com
•	iBatis PetStore: http://sourceforge.net/projects/ibatisjpetstore/
•	Links about common performance issues:
1. https://onyx.koli.ch/get/6448/3+-+Common+Web+Application+Errors.pdf
2. http://www.crn.com/slide-shows/cloud/231000374/10-most-common-causes-of-web-mobile-app-performance-issues.htm?pgno=1
3. www.tracelytics.com/blog/two-most-common-performance-mistakes/
4. http://gettingreal.37signals.com/toc.php
5. http://jonathanhui.com/top-j2ee-application-performance-problems

PENDING TO:
- Get from IBM presentations info from the adoption barriers (problem statement)
+ architecture!(proposed approach)
+ Merge the ideas from the file KeyIdeas_FuturePerformanceEngineering.txt DONE
